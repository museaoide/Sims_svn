% ----------------------------------------------------------------
% AMS-LaTeX Paper ************************************************
% **** -----------------------------------------------------------
\documentclass[12pt,reqno,oneside]{amsart}
%\usepackage{srcltx} % SRC Specials: DVI [Inverse] Search
\usepackage[left=3cm,top=3cm]{geometry}
\usepackage{graphicx}
\allowdisplaybreaks[3]
\usepackage{calc,xspace}
%----------------------------------------------------
\usepackage{dcolumn}
\usepackage{amsmath}
\newcolumntype{d}[1]{D{.}{.}{#1}}  %now ``dj'' in a tabular or array spec generates a .-centered column
%                                with j decimal places to the right of the . .
%---------------------------------------------
\usepackage{mathpazo}
%\usepackage{mathptmx}
\usepackage[T1]{fontenc}
\linespread{1.5}        % Palatino needs more leading
\usepackage[scaled]{helvet}
\usepackage{courier}
\normalfont
\usepackage{hyperref}
%\usepackage{natbib}
%\usepackage{color}
%\usepackage{url} %--> url{http://www.x}
%\renewcommand{\baselinestretch}{1.1}
% ----------------------------------------------------------------
\vfuzz2pt % Don't report over-full v-boxes if over-edge is small
\hfuzz2pt % Don't report over-full h-boxes if over-edge is small
% THEOREMS -------------------------------------------------------
\newtheorem{thm}{Theorem}[section]
\newtheorem{cor}[thm]{Corollary}
\newtheorem{lem}[thm]{Lemma}
\newtheorem{prop}[thm]{Proposition}
\theoremstyle{definition}
\newtheorem{defn}[thm]{Definition}
\theoremstyle{remark}
\newtheorem{rem}[thm]{Remark}
%\numberwithin{equation}{section}
% MATH -----------------------------------------------------------
% \divby command defined below produces nice-looking, right-sized "a/b" template.
\newcommand{\divby}[2]{#1 \mathord{\left/ \vphantom{#1 #2} \right.}
    \kern-\nulldelimiterspace #2}
\newcommand{\norm}[1]{\left\Vert#1\right\Vert}
\newcommand{\abs}[1]{\left\vert#1\right\vert}
\newcommand{\set}[1]{\left\{#1\right\}}
\newcommand{\Real}{\mathbb R}
\newcommand{\eps}{\varepsilon}
\newcommand{\To}{\xrightarrow}
%e.g.: \[X_t \To[t\to \infty]{q.m.} Z\]
%\newcommand{\BX}{\mathbf{B}(X)}
% \newcommand{\A}{\mathcal{A}}
\renewcommand{\thesection}{\Roman{section}}
% \newenvironment{fulltable}[1]{\noindent\begin{tabular*}{\textwidth}
%    {@{}@{\extracolsep{\fill}}#1@{}}}{\end{tabular*}}
\DeclareMathOperator{\Var}{Var}
\newcommand{\given}{\mathrel |}
\newcommand{\tcol}{\text{:}}
\newcounter{year}
\setcounter{year}{\year}
\renewcommand{\theyear}{\arabic{year}\xspace}
% ----------------------------------------------------------------
\begin{document}

\section{System of equations}

\begin{align}
\text{M policy}             &:& \dot r &= -\phi_0 \left[(r - \bar r ) - \phi_1 (\dot p - \bar\pi ) - \phi_2 \dot c \right]+ \eps_{m} \\
\text{IS}^\ast              &:& -\frac{\dot\lambda}{\lambda} &= a - \frac{\dot a}{a} - \dot p - \gamma - \theta + \xi_r \\
\text{term struct.}^\ast    &:& r &= a - \frac{\dot a}{a} \\
\text{Phillips curve}^\ast  &:& \ddot p &= \beta \left[ (\dot p - \bar\pi ) - \delta (c - \bar c )\right] - \xi_{pc} \\
\text{Govt. Budg. Cnstr.}   &:& \dot b &=  (a - \frac{\dot a}{a} -\dot p -\gamma) b - \tau \\
\text{Fiscal policy}        &:& \dot \tau &= \omega e^c \dot c + \eps_\tau \\
\text{LM}^\ast              &:& \lambda &= e^{-\sigma c-\tfrac{\psi}{2}(1-\sigma)\dot c^2}\Bigl(1+\psi\bigl((1-\sigma)\dot c^2 -[\theta+\psi\ddot c\dot c(1-\sigma)] \dot c + \ddot c \bigr)\Bigr) \\
\text{IS shock}             &:& \dot\xi_r &= -\rho_r\xi_r + \eps_r \\
\text{PC shock}             &:& \dot\xi_{pc} &= -\rho_{pc}\xi_{pc} + \eps_{pc}
\end{align}

Steady state: $\bar\lambda = e^{-\sigma \bar c}$, $\bar b = \bar\tau / \theta$, $\dot p = \bar\pi$, $\bar r = \bar a = \gamma + \theta + \bar\pi$. \\
Parameters: $\bar c$, $\bar \tau$, $\bar\pi$.

\section{Priors}

See Table 1. I set the priors identically across subsamples (full sample: 1954Q4 - 2005Q1, first half: 1954Q4 - 1979Q2, second half: 1983Q1 - 2005Q1) except for the priors of the shock variances.

\subsection{Policy inertia and interest rate mean-reversion parameter}
The policy inertia parameter $\phi_{0}$ has a distribution similar to a logit-transformed beta distribution. Its pdf is
\begin{gather}
  f(x;\alpha,\beta,m,s) = \frac{p^{-q}q^{-p}}{B(p,q)}\frac{1}{\frac{1}{\alpha}e^{\alpha \left(\frac{x-m}{s}\right)}+\frac{1}{\beta}e^{-\beta \left(\frac{x-m}{s}\right)}}I_{-\infty < x < \infty},
\end{gather}
where $p=\beta/(\alpha+\beta)$, $q=1-p$ for $\alpha, \beta > 0$ and $B(p,q)$ is Beta function. $m$ and $s$ are a location parameter and a scale parameter, respectively.

This distribution has a right-skewed distribution when $\alpha < \beta$. Let $Y$ is a random variable which has a $Beta(p,q)$ distribution. Then, $X$ can be obtained as
\begin{gather}
  X = \frac{1}{\alpha+\beta}\left( \log\frac{\alpha}{\beta} + \log \frac{Y}{1-Y} \right).
\end{gather}
There is a numerical problem in generating $X$ in this way. When $\alpha \ll \beta$, $Y$ is often numerically 1 or very close to 1 and it blows up $X$ to infinity. But with $\alpha = 1$, $\beta = 5$, $m=0$, and $s=1$, such chance is below .5\%. We ignore this problem.

\subsection{Shock variances}

Reflecting the result of the literature, I set priors for the shock variances (MP, IS, PC) so that they are slightly larger in the first half and smaller in the second half. For the full sample, I set in the middle of them. For the shock variance for primary surpluses, I set the same prior across the subsamples.

% need to do again, but current prior is set to put sigma_tau (primary surpluses) large and others small.
%The prior distribution for the variances of four shocks was chosen to match the magnitudes of the innovation variances in the transition equation with the magnitudes of the empirical counterparts\footnote{I used \textit{rfvar3.R} to compute the covariance matrix of the residuals of VAR. I used the entire sample (not presample), but we will be able to replace it with presample analysis, probably without having big differences.}. Since the innovation covariance matrix is nonlinear in model parameters as well as the shock variances ($\sigma^2_M$, $\sigma^2_{\tau}$, $\sigma^2_{r}$, $\sigma^2_{PC}$), I simulated the parameters from their prior distributions and then computed the implied innovation variances of the variables. Parameter draws which get non-existence or non-uniqueness from \textit{gensys} were discarded. The standard deviations of the residual covariance matrix from reduced-form VAR are 0.002, 0.003, 0.008, and 34.5 for $r,p,c$ and $\tau$.

\subsection{Parameters of gamma and inverse gamma distributions}

The first parameter and the second parameter of the gamma distribution are the shape and the rate (1/scale) parameter of the corresponding functions \textit{dgamma} and \textit{rgamma} of \textit{R}. The first and the second of the inverse gamma distribution are the shape parameter and the rate parameter, which is the scale parameter of the gamma distribution functions \textit{dgamma} and \textit{rgamma}. Therefore the reciprocal of the rate parameter of the inverse gamma distribution is the rate parameter of the gamma distribution.

\section{Penalty on parameters implying non-existence of solutions}

We use $csminwel$, the numerical optimization routine, to find the posterior density mode of the model. To improve the efficiency of numerical optimization, we adjust $div$ of $gensysct$ and impose a penalty on parameter values implying non-existence. Suppose that for given parameter values $gensysct$ returns $eu = (0,1)$, i.e. non-existence of solution. If there is one more unstable root than required for existence, then we adjust $div$ so that the smallest unstable root, say $r (>0)$, becomes stable: $div = r + \eps$ with a small positive number $\eps$ and we calculate the existence solution. But we add a penalty $A \times r^2$ to the log likelihood of the solution. Sometimes we have two identical unstable roots than required. In this case, we adjust $div$ in the same manner and compute a penalty as $2A \times r^2$. We don't deal with the case where we have three more unstable roots or more and we have two more, but different from each other, unstable roots. We set their log likelihoods as $-\infty$. Note that this penalty is so that the optimization routine moves smoothly around the boundary of existence. $A$ on a penalty is sufficiently large and the routine does not find a local maximum beyond the boundary.

\section{Measurement error on the real value of debt}

We include $b$, the real value of debt with measurement errors. That is, the observation of the debt, $\tilde b$, follows
\begin{gather}
\tilde b_{t} = b_{t} + \xi_{b,t}
\end{gather}
where the measurement error is an AR(1) process: $\xi_{b,t} = \xi_{b,t-1} + \eps_{t}$. This measurement error equation is added in the observation equation of the state space representation of the model. The measurement equation of the model, the linearized solution, includes only $b$.

\section{Miscellaneous}

\pagebreak
\begin{table}[b]
%\begin{center}
% use packages: array
\caption{Prior distributions of the parameters: discussed before}\label{tbl:prior1}
\begin{tabular}{lllll}
\hline
parameter           & distribution          & mean      & median    & std. \\
\hline \\
$\phi_0$            &                       &           & 0.2       &        \\
$\phi_1$            & $N(1,1)$              & 1         &           & 1      \\
$\phi_2$            & $N(.5,1.5^2)$         & .5        &           & 1.5    \\
$\gamma$            & $N(.005,.005^2)$      & .005      &           & .005   \\
$\rho$              & $Gamma(1.5,30)$       & .05       &           & .0408  \\
$\sigma$            & $Gamma(3,1)$          & 3         &           & 1.7321 \\
$\beta$             & $Gamma(1,20)$         & .05       &           & .05    \\
$\delta$            & $Gamma(1,1)$          & 1         &           & 1      \\
$\omega$            & $Beta(1,19)$          & .05       &           & .0476  \\
$\psi$              & $Gamma(1,1)$          & 1         &           & 1      \\
$\bar\pi$           & $N(.02,.04^2)$        & .02       &           & .04    \\
$\bar c$            & $N(c_{t=0},.1^2)$     & 7.634     &           & .1     \\
$\bar\tau$          & $N(.02\times b_{t=0},\cdot^2)$       & 16.6191   &           & 16.6191   \\
$\rho_r$            & $Gamma(1.5,1.5)$      & 2.25      &           & 1.8371 \\
$\rho_{pc}$         & $Gamma(1.5,1.5)$      & 2.25      &           & 1.8371 \\
$\sigma^2_m$        & $IG(1,.001)$          &           &           &        \\
$\sigma^2_\tau$     & $IG(1,.005)$          &           &           &        \\
$\sigma^2_r$        & $IG(1,.001)$          &           &           &        \\
$\sigma^2_{pc}$     & $IG(1,.001)$          &           &           &        \\
\hline
\end{tabular}
%\end{center}
\end{table}
\pagebreak

\begin{table}
%\begin{center}
% use packages: array
\caption{(NEW) Prior distributions of the parameters other than the variances}\label{tbl:prior2}
\begin{tabular}{lllll}
\hline
parameter           & distribution          & mean      & mode      & std. \\
\hline \\
$\phi_0$            & see the text          &           &           &        \\
$\phi_1$            & $N(1,1)$              & 1         &           & 1      \\
$\phi_2$            & $N(.5,1.5^2)$         & .5        &           & 1.5    \\
$\gamma$            & $N(.01,.005^2)$       & .01       &           & .005   \\
$\rho$              & $Gamma(1.5,30)$       & .05       &           & .0408  \\
$\sigma$            & $Gamma(3,1)$          & 3         &           & 1.7321 \\
$\beta$             & $Gamma(3,20)$         & .15       & .1        & .087   \\
$\delta$            & $Gamma(4,2)$          & 2         & 1.5       & 1      \\
$\omega$            & $Beta(1,1)$           & .5       &            &   \\
$\psi$              & $Gamma(1,1)$          & 1         &           & 1      \\
$\bar\pi$           & $N(.01,.005^2)$       & .01       &           & .005   \\
$\bar c$            & $N(c_{t=0},.1^2)$     & $c_{t=0}$ &           & .1     \\
$\bar\tau$          & $N(.02\times b_{t=0}/100,(\cdot/100)^2)$       &           &           &          \\
$\rho_r$            & $Gamma(1.5,1.5)$      & 2.25      &           & 1.8371 \\
$\rho_{pc}$         & $Gamma(1.5,1.5)$      & 2.25      &           & 1.8371 \\
$\rho_{b}$          & $Beta(8,2)$           & .8        &           & .12 \\
\hline
\end{tabular}
%\end{center}
\end{table}

\begin{table}[b]
%\begin{center}
% use packages: array
\caption{(NEW) Prior distributions of the variances}\label{tbl:prior3}
\begin{tabular}{lccc}
\hline
parameter           & Full sample      & First Half   & Second Half \\
\hline \\
$\sigma^2_m$        & $IG(2,5e+5)$     & $IG(2,1e+5)$          & $IG(2,1e+6)$       \\
$\sigma^2_\tau$     & $IG(2,1e+2)$     & $IG(2,1e+2)$          & $IG(2,1e+2)$       \\
$\sigma^2_r$        & $IG(2,5e+5)$     & $IG(2,1e+5)$          & $IG(2,1e+6)$       \\
$\sigma^2_{pc}$     & $IG(2,5e+5)$     & $IG(2,1e+5)$          & $IG(2,1e+6)$       \\
$\sigma^2_{b}$      & $IG(2,5e-3)$     & $IG(2,5e-3)$          & $IG(2,5e-3)$       \\
\hline
\end{tabular}
%\end{center}
\end{table}
\end{document}
% ---------------------------------------------------------------- 